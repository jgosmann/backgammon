\chapter{Netzwerkfunktionalität}
Funktionen zum Aufbau der Verbindung und Kommunikation mit einem Backgammon-Server werden von der Klasse \srcref[]{class }{NetBackgammonConnection}{ : public QTcpSocket;}{\wdir netbackgammon.h}{NetBackgammonConnection} bereitgestellt. Die Klasse führt Buch über den Verbindungsstatus, ob der User eingeloggt ist und er einem Spiel beigetreten ist. Zudem sendet sie für jede eingehende Nachricht vom Server das \name{Qt}-Signal \srcref{void NetBackgammonConnection::}{received_msg}{( NetBackgammonMsg msg );}{\wdir netbackgammon.h}{NetBackgammonConnection::received_msg(NetBackgammonMsg)}. Dabei wird mit einer Instanz der Klasse \srcref[]{class }{NetBackgammonMsg}{;}{\wdir netbackgammon.h}{NetBackgammonMsg} die Nachricht gespeichert. Diese Klasse bietet einen einfacheren Zugriff auf die Nachrichtenteile, so dass der Nachrichten-String nicht jedesmal neu aufgetrennt werden muss. Ein Empfänger des Signals ist unter anderen die Klasse \srcref[]{class }{ChatWidget}{ : public QWidget, public Ui::ChatWidget;}{\wdir gui/chatwidget.h}{ChatWidget}, welche eine Vielzahl der Nachrichten verarbeitet, um sie im Chat-Fenster auszugeben.

Für das eigentliche Backgammon-Spiel wird die Klasse \srcref[]{class }{NetBackgammon}{ : public BG::Backgammon;}{\wdir netbackgammon.h}{NetBackgammon} verwendet. Diese Klasse benötigt einen Zeiger auf eine Instanz der Klasse \srcref[]{class }{NetBackgammonConnection}{ : public QTcpSocket;}{\wdir netbackgammon.h}{NetBackgammonConnection} damit sie mit dem Backgammon-Server kommunizieren kann. Zudem muss (sofern es sich nicht um ein lokales Spiel handelt) \srcref[]{BG::Player NetBackgammon::}{m_net_player}{;}{\wdir netbackgammon.h}{NetBackgammon::m_net_player} auf den Spieler gesetzt werden, der über das Netzwerk gesteuert werden soll.

Jeweils wenn der nächste Spieler an die Reihe kommt, wird \srcref{void NetBackgammon::}{transmit_last_turn}{( void );}{\wdir netbackgammon.cpp}{void:NetBackgammon::transmit_last_turn(void)} aufgerufen. Diese Funktion prüft, ob der letzte Spieler der lokal gesteuerte ist, und sendet gegebenenfalls seine Züge mit der Funktion \srcref{void }{NetBackgammonConnection::turn}{( const BG::BackgammonTurn &turn );}{\wdir netbackgammon.cpp}{void:NetBackgammonConnection::turn(const:BG::BackgammonTurn&)}, welche die Züge vom internen Format in das des Servers konvertiert und diese so anordnet, dass sie vom Server akzeptiert werden, denn für die Klasse \srcref[]{class BG::}{Backgammon}{ : public QObject;}{\wdir backgammon.h}{BG::Backgammon} ist die Reihenfolge der Züge nicht von Bedeutung, so lange diese zusammen gültig sind.

Weiterhin verarbeitet die Klasse \srcref[]{class }{NetBackgammon}{ : public BG::Backgammon;}{\wdir netbackgammon.h}{NetBackgammon} eingehende Servernachrichten in der Funktion \srcref{void NetBackgammon::}{process_srv_msg}{( NetBackgammonMsg msg );}{\wdir netbackgammon.cpp}{void:NetBackgammon::process_srv_msg(NetBackgammonMsg)}. Insbesondere folgende Nachrichten werden verarbeitet:
\begin{itemize}
  \item \srvcmd{BOARD}: Übermittelt der Server das aktuelle Spielbrett, so wird dies in das klasseninterne Format konvertiert und gespeichert.
  \item \srvcmd{DICE} und \srvcmd{INFO:DICE}: Wenn die Würfel übermittelt werden, wird \srcref[]{short int BG::Backgammon::}{m_dice}{[ 4 ];}{\wdir backgammon.h}{BG::Backgammon::m_dice} entsprechend gesetzt.
  \item \srvcmd{INFO:TURN}: Wird ein Zug übermittelt, so wird dieser in das interne Format konvertiert, ausgeführt und in der Zugliste gespeichert. Zudem kommt der nächste Spieler an die Reihe.
  \item \srvcmd{ENDGAME}: Sollte das Backgammon-Spiel vorzeitig beendet werden, so werden Variablen der Klasse so gesetzt, dass das aktuelle Spiel auch programmintern beendet wird.
\end{itemize}
